\documentclass[11pt]{beamer}
\usetheme{Frankfurt}
\usepackage[utf8]{inputenc}
\usepackage[german]{babel}
\usepackage{amsmath}
\usepackage{amsfonts}
\usepackage{amssymb}
\usepackage{url}
\usepackage{color}
\definecolor{mygreen}{rgb}{0,0.6,0}
\definecolor{mygray}{rgb}{0.5,0.5,0.5}
\usepackage{listings}
% https://en.wikibooks.org/wiki/LaTeX/Source_Code_Listings
\lstset{basicstyle=\scriptsize % http://texblog.org/2012/08/29/changing-the-font-size-in-latex/
        , commentstyle=\color{mygreen}
        , keywordstyle=\color{blue}
        , language=C++
        %, frame=single
    }

\author{Richard Ulrich}
\title{C++11 initializer lists}
\subtitle{unified initialization of containers and other objects}
%\setbeamercovered{transparent} 
%\setbeamertemplate{navigation symbols}{} 
%\logo{cubx.png} 
\institute{cubx Software AG} 
%\date{} 
\subject{BORM developer day} 
\titlegraphic{\includegraphics[width=2cm]{cubx.png}}

\begin{document}

\begin{frame}
\titlepage
\end{frame}

%\begin{frame}
%\tableofcontents
%\end{frame}

\begin{frame}{container initialization in C++98}
\lstinputlisting[language=C++]{@initializerlists_SOURCE_DIR@/container_98.cpp}
\end{frame}

\begin{frame}{named initializer lists}
\lstinputlisting[language=C++]{@initializerlists_SOURCE_DIR@/named_initializerlists.cpp}
\end{frame}

\begin{frame}{container initialization in C++11}
\lstinputlisting[language=C++]{@initializerlists_SOURCE_DIR@/container_11.cpp}
\end{frame}

\begin{frame}{initialization list constructor}
\lstinputlisting[language=C++]{@initializerlists_SOURCE_DIR@/constructor.cpp}
\end{frame}

\begin{frame}{What the celebrities say}
\textbf{Herb Sutter} (\href{http://herbsutter.com/elements-of-modern-c-style}{herbsutter.com/elements-of-modern-c-style})\\

\emph{What hasn’t changed}: When initializing a local variable whose type is non-POD or auto, continue using the familiar = syntax without extra curly braces.\\
In other cases, including especially everywhere that you would have used ( ) parentheses when constructing an object, prefer using { } braces instead. Using braces avoids several potential problems: you can’t accidentally get narrowing conversions (e.g., float to int), you won’t occasionally accidentally have uninitialized POD member variables or arrays, and you’ll avoid the occasional C++98 surprise that your code compiles but actually declares a function rather than a variable because of a declaration ambiguity in C++’s grammar – what Scott Meyers famously calls “C++’s most vexing parse.” There’s nothing vexing about the new style.
\end{frame}

\begin{frame}{Herb Sutter code part 1}
\lstinputlisting[language=C++]{@initializerlists_SOURCE_DIR@/sutter1.cpp}
\end{frame}

\begin{frame}{Herb Sutter code part 2}
\lstinputlisting[language=C++]{@initializerlists_SOURCE_DIR@/sutter2.cpp}
\end{frame}

\begin{frame}{References and code}
All code presented in this document can be found at:\\
\href{https://github.com/ulrichard/experiments/tree/master/initializerlists}{https://github.com/ulrichard/experiments/tree/master/initializerlists}\\
\\[0.5cm]
It was written using vi, LaTeX and cmake to make sure the code actually compiles. For more details see:\\
\href{http://blog.ulrichard.ch/?p=1406}{http://blog.ulrichard.ch/?p=1406}
\end{frame}

\end{document}

