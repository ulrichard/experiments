\documentclass[11pt, aspectratio=169]{beamer}
%\setbeameroption{show notes}
\usetheme{CambridgeUS}
\usepackage[utf8]{inputenc}
\usepackage[german]{babel}
\usepackage{amsmath}
\usepackage{amsfonts}
\usepackage{amssymb}
\usepackage{url}
\usepackage{multimedia}
%\pgfpageuselayout{4 0n 1 with notes}[a4paper,border shrink=5mm]
\usepackage{color}
\definecolor{mygreen}{rgb}{0,0.6,0}
\definecolor{mygray}{rgb}{0.5,0.5,0.5}
\usepackage{listings}
\usepackage{outlines}
% https://en.wikibooks.org/wiki/LaTeX/Source_Code_Listings
% http://texblog.org/2012/08/29/changing-the-font-size-in-latex/
\lstset{basicstyle=\scriptsize 
        , commentstyle=\color{mygreen}
        , keywordstyle=\color{blue}
        , language=C++
        %, frame=single
    }

\author{Richard Ulrich}
\title{Docker}
\subtitle{An introduction to docker containers}
\setbeamercovered{dynamic} 
\institute{BORM Informatik AG} 
%\date{} 
\subject{BORM developer day} 
\titlegraphic{\includegraphics[width=2cm]{borm_logo.jpg}}

\begin{document}

%%%%%%%%%%%%%%%%%%%%%%%%%%%%%%%%%%%%%%%%%%%%%%%%%%%%%%%%%%%%%%%%%%%%%%%%%%%%%%%
\begin{frame}
\titlepage
\end{frame}

%%%%%%%%%%%%%%%%%%%%%%%%%%%%%%%%%%%%%%%%%%%%%%%%%%%%%%%%%%%%%%%%%%%%%%%%%%%%%%%
\begin{frame}{Why ``works on my machine'' is not enough}
\textbf{Releasing binaries reproducibly}
\begin{itemize}
\item distant past: release compiled binary from a developers machine
\item present: binaries from bamboo
\item source control system revisions
\item versions of external dependencies packages
\item configuration of the build agents
\item docker offers a way to define the environment in a text file
\item docker files can be versioned
\end{itemize}
\end{frame}

%%%%%%%%%%%%%%%%%%%%%%%%%%%%%%%%%%%%%%%%%%%%%%%%%%%%%%%%%%%%%%%%%%%%%%%%%%%%%%%
\begin{frame}{Table of contents}
\begin{outline}
\1 Introduction video
\1 Dissection of an example
  \2 Dockerfile
  \2 Build the container
  \2 Run the container
\end{outline}
\end{frame}

%%%%%%%%%%%%%%%%%%%%%%%%%%%%%%%%%%%%%%%%%%%%%%%%%%%%%%%%%%%%%%%%%%%%%%%%%%%%%%%
\begin{frame}{Introduction video}
%https://tex.stackexchange.com/questions/89088/how-to-embed-video-and-animation-in-latex-and-latex-beamer-step-by-step
\centering
\movie[label=show3,width=1.0\textwidth,poster
       ,autostart,showcontrols,loop] 
  {\includegraphics[width=1.0\textwidth]{pGYAg7TMmp0.png}}{pGYAg7TMmp0.webm}
\url{https://www.youtube.com/watch?v=pGYAg7TMmp0}
\end{frame}

%%%%%%%%%%%%%%%%%%%%%%%%%%%%%%%%%%%%%%%%%%%%%%%%%%%%%%%%%%%%%%%%%%%%%%%%%%%%%%%
\begin{frame}{Dissection of an example}
\begin{outline}
\1 Jaxx is a multi cryptocurrency wallet
\1 Unfortunately it is available only as closed source
\1 Risks include:
  \2 Stealing of private keys stored on the harddisk
  \2 Exposing transaction history from stored wallets
  \2 Polluting the environment with libraries in non standard locations
\1 Docker offers protection by running in an encapsulated environment
\1 Much more lightweight and convenient than a virtual machine
\end{outline}
\end{frame}

%%%%%%%%%%%%%%%%%%%%%%%%%%%%%%%%%%%%%%%%%%%%%%%%%%%%%%%%%%%%%%%%%%%%%%%%%%%%%%% 
\begin{frame}{Dockerfile - Defining the base environment} 
\lstinputlisting[language=bash, firstline=0, lastline=16]{@docker_SOURCE_DIR@/Dockerfile}
\end{frame}

%%%%%%%%%%%%%%%%%%%%%%%%%%%%%%%%%%%%%%%%%%%%%%%%%%%%%%%%%%%%%%%%%%%%%%%%%%%%%%% 
\begin{frame}{Dockerfile - Download and unpack binary} 
\lstinputlisting[language=bash, firstline=17, lastline=21]{@docker_SOURCE_DIR@/Dockerfile}
\end{frame}

%%%%%%%%%%%%%%%%%%%%%%%%%%%%%%%%%%%%%%%%%%%%%%%%%%%%%%%%%%%%%%%%%%%%%%%%%%%%%%% 
\begin{frame}{Dockerfile - Determine and satisfy dependencies} 
\lstinputlisting[language=bash, firstline=23, lastline=33]{@docker_SOURCE_DIR@/Dockerfile}
\end{frame}

%%%%%%%%%%%%%%%%%%%%%%%%%%%%%%%%%%%%%%%%%%%%%%%%%%%%%%%%%%%%%%%%%%%%%%%%%%%%%%% 
\begin{frame}{Dockerfile - Set up user environment} 
\lstinputlisting[language=bash, firstline=35, lastline=49]{@docker_SOURCE_DIR@/Dockerfile}
\end{frame}

%%%%%%%%%%%%%%%%%%%%%%%%%%%%%%%%%%%%%%%%%%%%%%%%%%%%%%%%%%%%%%%%%%%%%%%%%%%%%%% 
\begin{frame}{Dockerfile - Entry point} 
\lstinputlisting[language=bash, firstline=51, lastline=52]{@docker_SOURCE_DIR@/Dockerfile}
\end{frame}

%%%%%%%%%%%%%%%%%%%%%%%%%%%%%%%%%%%%%%%%%%%%%%%%%%%%%%%%%%%%%%%%%%%%%%%%%%%%%%% 
\begin{frame}{Execution} 
Build the docker container, and execute the bundled application
\lstinputlisting[language=bash]{@docker_SOURCE_DIR@/jaxx.sh}
\end{frame}

%%%%%%%%%%%%%%%%%%%%%%%%%%%%%%%%%%%%%%%%%%%%%%%%%%%%%%%%%%%%%%%%%%%%%%%%%%%%%%% 
\begin{frame}{Docker on Windows} 
\begin{itemize}
\item Same principle as on linux
\item More complicated to install stuff
\item See for yourself: \url{https://blogs.msdn.microsoft.com/heaths/2017/09/18/installing-build-tools-for-visual-studio-2017-in-a-docker-container/}
\end{itemize}
\end{frame}

%%%%%%%%%%%%%%%%%%%%%%%%%%%%%%%%%%%%%%%%%%%%%%%%%%%%%%%%%%%%%%%%%%%%%%%%%%%%%%%
\begin{frame}{Closing}
All code presented in this document can be found at:\\
\url{https://github.com/ulrichard/experiments/tree/master/docker}\\
-\\
It was written using vim, LaTeX and cmake. For more details see:\\
\url{https://ulrichard.ch/blog/?p=1406}\\
-\\
Thanks for listening
-\\
Do you have questions?
\end{frame}

\end{document}

